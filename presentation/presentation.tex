% !TeX program = pdflatex
\documentclass{beamer}
\usepackage[utf8]{inputenc}
\usepackage[polish]{babel}
\usepackage[T1]{fontenc}
\usepackage{tikz}
\usetikzlibrary{positioning}
\usepackage{graphicx}
\usepackage{hyperref}
\usepackage{biblatex}
\addbibresource{references.bib}

% Ulepszony motyw i kolory
\usetheme{Madrid}
\usecolortheme{dolphin}
\setbeamertemplate{navigation symbols}{}
\setbeamertemplate{footline}[frame number]

% Lepsze style dla diagramów
\tikzset{
  block/.style={
    draw,
    rectangle,
    rounded corners,
    minimum width=2cm,
    minimum height=1cm,
    align=center,
    fill=blue!10
  }
}

\title{Generowanie cyfr pisanych odręcznie na bazie sieci neuronowej o architekturze autoencodera}
\subtitle{Analiza i implementacja różnych modeli generatywnych}
\author{Prowadzący: Adam Świtoński}
\institute{Politechnika Śląska}
\date{Maj 2024}

\begin{document}

\begin{frame}
  \titlepage
\end{frame}

\begin{frame}{Plan prezentacji}
  \tableofcontents
\end{frame}

\begin{frame}{Opis projektu}
  \begin{block}{Cel projektu}
    Zastosowanie sieci neuronowej o architekturze autoencodera do generowania niskorozdzielczych obrazów cyfr pisanych odręcznie.
  \end{block}
  
  \begin{itemize}
    \item Trenowanie różnych wariantów autoencoder'a
    \item Generowanie nowych cyfr poprzez podawanie losowych wartości na wejście dekodera
    \item Badanie wpływu struktury sieci (liczba warstw, liczba neuronów) oraz parametrów uczenia
    \item Wykorzystanie bazy danych MNIST
  \end{itemize}
  
  \begin{block}{Rozważane warianty}
    Klasyczny autoencoder, wariacyjny autoencoder (VAE), GAN, Diffusion, VQ-VAE, Conditional VAE
  \end{block}
\end{frame}

% --- MODELE ---

\section{Modele}

% AUTOENCODER
\begin{frame}{Autoencoder}
  \begin{columns}
    \begin{column}{0.6\textwidth}
      \textbf{Autoencoder} to rodzaj sieci neuronowej, która uczy się kompresować dane wejściowe do reprezentacji o niższym wymiarze (kod), a następnie rekonstruować oryginalne dane z tej reprezentacji.
      
      \medskip
      \textbf{Zastosowania:}
      \begin{itemize}
        \item Redukcja wymiarowości
        \item Denoising (odszumianie)
        \item Generowanie nowych danych
      \end{itemize}
      
      \textbf{Zalety:} prostota, szybki trening, interpretowalność
      
      \textbf{Wady:} ograniczona zdolność generatywna, brak kontroli nad rozkładem latentnym
      
      \textbf{Bibliografia:} \cite{hinton2006reducing}
    \end{column}
    \begin{column}{0.4\textwidth}
      \begin{tikzpicture}[scale=0.6, transform shape]
        \node[block] (input) at (0,0) {Wejście};
        \node[block, below=0.8cm of input] (enc) {Enkoder};
        \node[block, below=0.8cm of enc] (code) {Kod};
        \node[block, below=0.8cm of code] (dec) {Dekoder};
        \node[block, below=0.8cm of dec] (output) {Wyjście};
        
        \draw[->, thick] (input) -- (enc);
        \draw[->, thick] (enc) -- (code);
        \draw[->, thick] (code) -- (dec);
        \draw[->, thick] (dec) -- (output);
      \end{tikzpicture}
    \end{column}
  \end{columns}
\end{frame}

% VAE
\begin{frame}{Wariacyjny Autoencoder (VAE)}
  \begin{columns}
    \begin{column}{0.6\textwidth}
      \textbf{VAE} to probabilistyczne rozszerzenie autoencodera, które modeluje rozkład latentny danych.
      
      \medskip
      \textbf{Kluczowe cechy:}
      \begin{itemize}
        \item Modelowanie rozkładu latentnego ($\mu$, $\sigma$)
        \item Regularyzacja poprzez KL-dywergencję
        \item Generowanie nowych próbek przez próbkowanie
      \end{itemize}
      
      \textbf{Zalety:} generatywność, ciągła przestrzeń latentna, możliwość interpolacji
      
      \textbf{Wady:} rozmyte próbki, trudność w trenowaniu
      
      \textbf{Bibliografia:} \cite{kingma2013auto}
    \end{column}
    \begin{column}{0.4\textwidth}
      \begin{tikzpicture}[scale=0.6, transform shape]
        \node[block] (input) at (0,0) {Wejście};
        \node[block, below=0.8cm of input] (enc) {Enkoder};
        \node[block, below=0.8cm of enc] (mu) {$\mu$, $\sigma$};
        \node[block, below=0.8cm of mu] (dec) {Dekoder};
        \node[block, below=0.8cm of dec] (output) {Wyjście};
        
        \draw[->, thick] (input) -- (enc);
        \draw[->, thick] (enc) -- (mu);
        \draw[->, thick] (mu) -- (dec);
        \draw[->, thick] (dec) -- (output);
      \end{tikzpicture}
    \end{column}
  \end{columns}
\end{frame}

% CONDITIONAL VAE
\begin{frame}{Conditional VAE}
  \begin{columns}
    \begin{column}{0.6\textwidth}
      \textbf{Conditional VAE} (CVAE) to wariacyjny autoencoder, który dodatkowo warunkuje generowanie na zadanej klasie (np. cyfra).
      
      \medskip
      \textbf{Kluczowe cechy:}
      \begin{itemize}
        \item Warunkowanie generowania na etykietach
        \item Możliwość sterowania procesem generacji
        \item Łączenie etykiet z danymi wejściowymi
      \end{itemize}
      
      \textbf{Zalety:} kontrola nad generowanymi danymi, elastyczność
      
      \textbf{Wady:} większa złożoność, wymaga etykiet
      
      \textbf{Bibliografia:} \cite{sohn2015learning}
    \end{column}
    \begin{column}{0.4\textwidth}
      \begin{tikzpicture}[scale=0.6, transform shape]
        \node[block] (input) at (0,0) {Wejście + klasa};
        \node[block, below=0.8cm of input] (enc) {Enkoder};
        \node[block, below=0.8cm of enc] (mu) {$\mu$, $\sigma$};
        \node[block, below=0.8cm of mu] (dec) {Dekoder};
        \node[block, below=0.8cm of dec] (output) {Wyjście};
        
        \draw[->, thick] (input) -- (enc);
        \draw[->, thick] (enc) -- (mu);
        \draw[->, thick] (mu) -- (dec);
        \draw[->, thick] (dec) -- (output);
      \end{tikzpicture}
    \end{column}
  \end{columns}
\end{frame}

% VQ-VAE
\begin{frame}{VQ-VAE}
  \begin{columns}
    \begin{column}{0.6\textwidth}
      \textbf{VQ-VAE} to autoencoder, w którym przestrzeń latentna jest kwantyzowana do skończonego zbioru wektorów (słownik kodów).
      
      \medskip
      \textbf{Kluczowe cechy:}
      \begin{itemize}
        \item Dyskretna przestrzeń latentna
        \item Kwantyzacja wektorowa
        \item Słownik kodowy
      \end{itemize}
      
      \textbf{Zalety:} dyskretna reprezentacja, dobre wyniki w generowaniu sekwencji
      
      \textbf{Wady:} trudność w trenowaniu, konieczność doboru rozmiaru słownika
      
      \textbf{Bibliografia:} \cite{van2017neural}
    \end{column}
    \begin{column}{0.4\textwidth}
      \begin{tikzpicture}[scale=0.6, transform shape]
        \node[block] (input) at (0,0) {Wejście};
        \node[block, below=0.8cm of input] (enc) {Enkoder};
        \node[block, below=0.8cm of enc] (quant) {Kwantyzacja};
        \node[block, below=0.8cm of quant] (dec) {Dekoder};
        \node[block, below=0.8cm of dec] (output) {Wyjście};
        
        \draw[->, thick] (input) -- (enc);
        \draw[->, thick] (enc) -- (quant);
        \draw[->, thick] (quant) -- (dec);
        \draw[->, thick] (dec) -- (output);
      \end{tikzpicture}
    \end{column}
  \end{columns}
\end{frame}

% GAN
\begin{frame}{Generative Adversarial Network (GAN)}
  \begin{columns}
    \begin{column}{0.6\textwidth}
      \textbf{GAN} to model generatywny składający się z dwóch sieci: generatora (tworzy próbki) i dyskryminatora (odróżnia próbki prawdziwe od fałszywych).
      
      \medskip
      \textbf{Kluczowe cechy:}
      \begin{itemize}
        \item Układ rywalizujący (gra dwuosobowa)
        \item Generator tworzy coraz lepsze próbki
        \item Dyskryminator staje się coraz trudniejszy do oszukania
      \end{itemize}
      
      \textbf{Zalety:} realistyczne próbki, duża elastyczność
      
      \textbf{Wady:} trudność w trenowaniu, niestabilność, mode collapse
      
      \textbf{Bibliografia:} \cite{goodfellow2014generative}
    \end{column}
    \begin{column}{0.4\textwidth}
      \begin{tikzpicture}[scale=0.6, transform shape]
        \node[block] (noise) at (0,0) {Szum};
        \node[block, below=0.8cm of noise] (gen) {Generator};
        \node[block, below=0.8cm of gen] (img) {Obraz};
        \node[block, below=0.8cm of img] (disc) {Dyskryminator};
        \node[block, below=0.8cm of disc] (result) {Prawdziwy/Fałszywy};
        
        \draw[->, thick] (noise) -- (gen);
        \draw[->, thick] (gen) -- (img);
        \draw[->, thick] (img) -- (disc);
        \draw[->, thick] (disc) -- (result);
      \end{tikzpicture}
    \end{column}
  \end{columns}
\end{frame}

% DIFFUSION
\begin{frame}{Diffusion Model}
  \begin{columns}
    \begin{column}{0.6\textwidth}
      \textbf{Model dyfuzji} to nowoczesny model generatywny, który uczy się odszumiania danych przez odwracanie procesu stopniowego dodawania szumu.
      
      \medskip
      \textbf{Kluczowe cechy:}
      \begin{itemize}
        \item Proces forward (dodawanie szumu)
        \item Proces reverse (przewidywanie i usuwanie szumu)
        \item Iteracyjne próbkowanie
      \end{itemize}
      
      \textbf{Zalety:} wysoka jakość generowanych próbek, stabilność treningu
      
      \textbf{Wady:} długi czas generowania, złożoność obliczeniowa
      
      \textbf{Bibliografia:} \cite{ho2020denoising}
    \end{column}
    \begin{column}{0.4\textwidth}
      \begin{tikzpicture}[scale=0.55, transform shape]
        \node[block] (xt) at (0,0) {$x_T$ (szum)};
        \node[block, below=0.6cm of xt] (dots1) {$\cdots$};
        \node[block, below=0.6cm of dots1] (x1) {$x_1$};
        \node[block, below=0.6cm of x1] (x0) {$x_0$ (obraz)};
        
        \draw[->, thick] (xt) -- (dots1) node[midway, right] {Odszumianie};
        \draw[->, thick] (dots1) -- (x1);
        \draw[->, thick] (x1) -- (x0);
        \draw[->, thick] (x0) to [out=180,in=180] node[midway, left] {Dodawanie szumu} (xt);
      \end{tikzpicture}
    \end{column}
  \end{columns}
\end{frame}

% --- PORÓWNANIE MODELI ---

\section{Porównanie modeli}

\begin{frame}{Porównanie modeli generatywnych}
  \begin{tabular}{|p{2.5cm}|p{3.5cm}|p{3.5cm}|}
    \hline
    \textbf{Model} & \textbf{Zalety} & \textbf{Wady} \\
    \hline
    Autoencoder (2006) & Prostota implementacji, szybki trening & Słaba generatywność, rozmyte obrazy \\
    \hline
    VAE (2013) & Solidne podstawy teoretyczne, ciągła przestrzeń latentna & Rozmyte obrazy, trudność balansowania rekonstrukcji i KL divergencji \\
    \hline
    GAN (2014) & Ostre, realistyczne próbki & Niestabilność treningu, mode collapse \\
    \hline
    Conditional VAE (2015) & Kontrola nad procesem generacji, warunkowanie na klasach & Większa złożoność implementacji, wymaga etykiet \\
    \hline
    VQ-VAE (2017) & Ostrzejsze obrazy, dobra kompresja & Trudniejszy do trenowania, problemy z kwantyzacją \\
    \hline
    Diffusion (2020) & Najlepsza jakość obrazów, stabilny trening & Powolne próbkowanie, wysoka złożoność obliczeniowa \\
    \hline
  \end{tabular}
\end{frame}

% --- WYZWANIA I PRZYKŁADY ---

\section{Wyzwania implementacyjne}

\begin{frame}{Wyzwania implementacyjne}
  \begin{itemize}
    \item \textbf{Dobór architektury:} Liczba warstw, liczba neuronów, funkcje aktywacji
    \item \textbf{Dobór wymiarowości przestrzeni latentnej:} Zbyt mała - utrata informacji, zbyt duża - brak generalizacji
    \item \textbf{Balansowanie funkcji straty:} Np. w VAE balans między rekonstrukcją a regularyzacją KL
    \item \textbf{Stabilność treningu:} Szczególnie w przypadku GAN-ów
    \item \textbf{Efektywność obliczeniowa:} Modele dyfuzji wymagają wielu kroków podczas generowania
    \item \textbf{Ocena jakości wygenerowanych próbek:} Metody ilościowe vs jakościowe
  \end{itemize}
\end{frame}

\section{Przykłady zastosowań}

\begin{frame}{Przykłady zastosowań}
  \begin{columns}
    \begin{column}{0.5\textwidth}
      \textbf{Generowanie danych syntetycznych:}
      \begin{itemize}
        \item Augmentacja danych w uczeniu maszynowym
        \item Syntetyczne dane dla trenowania innych modeli
        \item Generowanie przykładów do zastosowań edukacyjnych
      \end{itemize}
    \end{column}
    \begin{column}{0.5\textwidth}
      \textbf{Zastosowania praktyczne:}
      \begin{itemize}
        \item Transfer stylu pisma
        \item Uzupełnianie brakujących fragmentów
        \item Korekta i poprawa pisma odręcznego
        \item Konwersja cyfr między różnymi stylami
      \end{itemize}
    \end{column}
  \end{columns}
\end{frame}

% --- BIBLIOGRAFIA ---

\begin{frame}[allowframebreaks]{Bibliografia}
  \printbibliography
\end{frame}

\end{document}